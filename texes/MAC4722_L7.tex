% to change the appearance of the header, questions, problems or subproblems, see the homework.cls file or
% override the \Problem, \Subproblem, \question or \printtitle commands.

% The hidequestions option hides the questions. Remove it to print the questions in the text.

\title{MAC4722 - Lista 7}

\documentclass{homework}

\usepackage[utf8]{inputenc}
\usepackage{amsthm}
\usepackage{mathtools}

\usepackage{tikz}
\usetikzlibrary{automata,positioning}

\usepackage{graphicx}
\graphicspath{ {images/} }

\newtheorem*{theorem}{Lema do Bombeamento para Linguagens Livres de Contexto}

% Set up your name, the course name and the homework set number.
\homeworksetup{
    username={Jean Fobe, N$^o$USP 7630573},
    course={Linguagens, Autômatos e Computabilidade - MAC4722},
    setnumber=7}
\begin{document}% this also prints the header.
\pagestyle{fancy}
\fancyfoot[L]{Jean Fobe, 7630573}

% use starred problems or subproblems to apply manual numbering.
\problem*{1}
\question{Use o Lema do bombeamento para mostrar que as seguintes linguagens não são livres de contexto.}
\begin{theorem}
	Se uma linguagem L é livre de contexto, então existe algum inteiro $p \geq 1$ tal que qualquer cadeia s em L com $| s | \geq p$ (onde p é um "comprimento de bombeamento") pode ser escrita como
    \[s = uvxyz,\]
com as subcadeias u, v, x, y e z, de tal modo que
	\begin{enumerate}
		\item $p \geq |vxy|$,
    		\item $|vy| \geq 1$, e
    		\item $uv^i xy^i z$ pertence a L para todo $i \geq 0$. 	
	\end{enumerate}
\end{theorem}
	\begin{itemize}
		\item[(a)] $L = \{0^n 1^n 0^n 1^n:\ n \geq 0\}$\\
		$Resp:$ Vamos provar por contradição que $L$ não é livre de contexto.\\
			Seja $p$ o inteiro fornecido pelo \textbf{Lema do Bombeamento para Linguagens Livres de Contexto} para $L$ e a palavra $s \in \L, s = 0^p 1^p 0^p 1^{p - r} 1^r$. Com $x = 0^p 1^p 0^p 1^{p - r}, y = 1^r, u = v = z = \lambda, p \geq |vxy| \geq |vy| \geq 1\ e\ p > r$, considere a palavra ($i = 0$):
			\[t = uv^0xy^0z\ = 0^p 1^p 0^p 1^{p - r} (1^r)^0 = 0^p 1^p 0^p 1^{p - r}\]
			Como o sufixo de apenas '1's de $t$ tem tamanho menor que $p$, $t \notin L$. Isso contradiz a condição \textit{3} do \textbf{LBLLC}.\\
			Então, $L$ não é linguagem livre de contexto também.
		\item[(b)] $L = \{t_1\# t_2\# ...\# t_k:\ k \geq 2,\ cada\ t_i \in \{a,b\}^*,\ e\ para\ algum\ i \neq j, t_i = t_j\}$\\
		$Resp:$ Tentaremos provar por contradição que $L$ não é livre de contexto.\\
		Seja $p$ o inteiro fornecido pelo \textbf{LBLLC} para $L$ e a palavra $s \in \L, s = b^p\#b^{p - r}b^r$. Com $u = b^p, x = \#b^{p - r}, y = b^r, v = z = \lambda, p \geq |vxy| \geq |vy| \geq 1\ e\ p > r$, considere a palavra ($i = 0$):
			\[t = uv^0xy^0z\ = b^p\#b^{p - r}(b^r)^0 = b^p\#b^{p - r}\]
			Como o sufixo $t_s$ e o prefixo $t_p$ separados por $\#$ na palavra $t$ são diferentes, $t_s \neq t_p\ e\ t \notin L$, o que contradiz o item \textit{3} do \textbf{LBLLC}.\\
			Portanto, $L$ não é livre de contexto.
		\end{itemize}

\pagebreak

\problem*{2}
\question{Para quaisquer linguagens $A$ e $B$, defina $A \diamond B = \{xy: x \in A, y \in B\ e\ |x|=|y|\}$. Mostre que, se $A$ e $B$ são regulares, então $A \diamond B$ é uma linguagem livre de contexto.}
	$Resp:$ Vamos definir uma linguagem livre de contexto $C = \{wz: w,z \in \Sigma^*\ e\ |w|=|z|\}$. Sabendo que existem as linguagens regulares $D,E \subseteq \Sigma^*$, observamos que:
	\begin{align}
		& D.E = F,\ uma\ linguagem\ regular;\\
		& F \cap C = G,\ linguagem\ livre\ de\ contexto;\\
		& G = C =\{wz: w,z \in \Sigma^*\ e\ |w|=|z|\}.
	\end{align}
	Podemos colocar informalmente que $C$ é livre de contexto tendo em mente uma gramática que gera palavras com prefixo e sufixo de mesmo tamanho pertencendo $\Sigma^*$ (Um conjunto de regras possíveis seria $R = \{S \rightarrow ASA\ |\ lambda,\ A \rightarrow \sigma\}, \sigma \in \Sigma$). A partir desses argumentos, para $A$ e $B$  sendo linguagens regulares, $A = \Sigma^* \cap A$ e $B = \Sigma^* \cap B$, teremos $A \diamond B = A.B \cap C$, uma linguagem livre de contexto.
	
\problem*{3}
\question{Seja $L = \{x\#y: x,y \in \{0,1\}^*\ e\ x \neq y\}$. Mostre que L é uma linguagem livre de contexto}
	$Resp:$ Para mostrarmos que $L$ é livre de contexto vamos definir uma gramática $G = \{V, \Sigma, R, S\}$, de modo que $L(G) = L$.
	\begin{align*}
		V & = \{S,A\};\\
		\Sigma & = \{0,1\};\\
		R & = \{S \rightarrow ASA\ |\ 0\#1\ |\ 1\#0\ |\ \#0\ |\ \#1\ |\ 0\#\ |\ 1\#,\\
			& \qquad A \rightarrow 0\ |\ 1\};\\
		S & = S;
	\end{align*}
	Agora observando as palavras que as regras de $R$ geram:
		\begin{align*}
			\forall x \in \Sigma^*,\ A & \xRightarrow[\text{G}]{*} x \iff x \in \{0,1\};\\
			\forall w \in \Sigma^*,\ S & \xRightarrow[\text{G}]{*} w \iff w \in \{0,1\}^*.\{0\#1,1\#0,\#0,\#1,0\#,1\#\}.\{0,1\}^*;\\
			w \in \{0,1\}^*.\{0\#1,1\#0,\#0,\#1,0\#,1\#\}.\{&0,1\}^* \iff w = x\#y,\ x,y \in \{0,1\}^*\ e\ x \neq y;
		\end{align*}
		Dessa forma justificamos que $G$ gera a linguagem acima.
		
\pagebreak		
		
\problem*{4}
\question{Considere a gramática livre de contexto $G = \{V, \Sigma, R, S\}$, onde $V = \{S\},\ \Sigma = \{a,b\},\ e\\ R = \{S \rightarrow aSbS\ |\ bSaS\ |\ \lambda\}$.}
	\begin{enumerate}
		\item[a)] $G$ é ambígua? Justifique sua resposta.\\
		Sabemos que $G$ será ambígua se pelo menos duas derivações diferentes resultarem numa mesma palavra. Com isso, observamos que podemos derivar a palavra $abab$ das seguintes formas:
		\begin{align*}
			S & \xRightarrow[\text{G}]{aSbS} aSbS \xRightarrow[\text{G}]{aSbS} aSb(aSbS) \xRightarrow[\text{G}]{\lambda} abaSbS \xRightarrow[\text{G}]{\lambda} ababS \xRightarrow[\text{G}]{\lambda} abab;\\
			S & \xRightarrow[\text{G}]{aSbS} aSbS \xRightarrow[\text{G}]{bSaS} a(bSaS)bS \xRightarrow[\text{G}]{\lambda} abaSbS \xRightarrow[\text{G}]{\lambda} ababS \xRightarrow[\text{G}]{\lambda} abab.
		\end{align*}
		Temos, então, que $G$ é ambígua.
		\item[b)] Prove que $L(G) = \{w \in \Sigma^*: |w_a| = |w_b|\}$.
			\begin{itemize}
				\item $L \subseteq L(G)$:\\
				Vamos provar por indução  no tamanho da palavra derivada. 
					\begin{itemize}
						\item[Base:] $w_a = w_b = \lambda,\ S \xRightarrow[\text{G}]{} \lambda$, pois $(S \rightarrow \lambda) \in R$;
						\item[Hipótese:]  $|w| \geq 0,\ S \xRightarrow[\text{G}]{*} w, |w|_a=|w|_b \leq n, n \geq 0$;
						\item[Passo:] Observamos que todas as derivações de $S$ tem números iguais de 'a's e 'b's, uma vez que, ou $S$ introduz, um terminal 'a' e um terminal 'b' na cadeia, ou $S$ deriva a palavra vazia. Então $S \xRightarrow[\text{G}]{2n-1} w, |w|_a=|w|_b \leq 2n - 1$;
						\item Como para $S \xRightarrow[\text{G}]{n} w, 0 \leq |w|_a=|w|_b \leq n$, $L \subseteq L(G)$;
					\end{itemize}
				\item $L(G) \subseteq L$:\\
				Vamos provar por indução no número de passos de derivação.
					\begin{itemize}
						\item[Base:] $S \xRightarrow[\text{G}]{1} x, x = \lambda$, pois essa é a única derivação que resulta em palavra composta por apenas terminais e $|w|_a = |w|_ b = 0$;
						\item[Hipótese:]  $n \geq 1,\ S \xRightarrow[\text{G}]{2n -1} w, |w|_a=|w|_b \leq n, n \geq 0$;
						\item[Passo:] Vamos supor $S \xRightarrow[\text{G}]{2n+1} w, |w|_a=|w|_b \leq n + 1$, então, $S \xRightarrow[\text{G}]{} aSbS \xRightarrow[\text{G}]{*} w$ e $S \xRightarrow[\text{G}]{} bSaS \xRightarrow[\text{G}]{*} w$.\\
						Logo, $\exists x,y \in \Sigma^*$ tal que $w = ayb = bxa$ e $S \xRightarrow[\text{G}]{2n - 3} y,\ S \xRightarrow[\text{G}]{2n - 3} x$. Pela hipótese de indução e o conjunto de regras da gramática, segue que $|y|_a = |y|_b = |x|_a = |x|_b \leq 2n - 1$.
						\item Então, $L(G) \subseteq L$;
					\end{itemize}
			\end{itemize}
	\end{enumerate}

\pagebreak

\problem*{5}
\question{Escreva um descrição informal e formal de uma máquina de Turing que decida a linguagem $L = \{ww:w \in \{0,1\}^*\}$}
	\begin{itemize}
		\item Formalmente, temos a Tupla $M = (Q,\Sigma,\Gamma,s,b,F,\delta)$:
		\begin{align*}
			Q = &\ \{s, q_{aceita}, q_{rejeita}, q_{1}, q{2}, q_{3}, q_{4}, q_{5}, q_{6}, q_{7}, q_{8}, q_{9}, q_{10}, q_{11}, q_{12}, q_{13}, q_{14}, q_{15}, q_{16}, q_{17}, q_{18}\};\\
			\Sigma = &\ \{0,1\};\\
			\Gamma = &\ \{0,1,\_\};\\
			s = &\ s;\\
			b = &\ \_;\\
			F = &\ \{q_{aceita}, q_{rejeita}\};\\
			\delta :\ &\ (Q - F) \times \Gamma \rightarrow Q \times \Gamma \times \{D,E\};\\
			&\qquad \delta(s,0) \rightarrow (q_1,0,D);
			\qquad \delta(s,1) \rightarrow (q_1,1,D);\\
			&\qquad \delta(s,\_) \rightarrow (q_{aceita},\_,E);\\
			&\qquad \delta(q_1,1) \rightarrow (q_2,1,D);
			\qquad \delta(q_1,0) \rightarrow (q_2,0,D);\\
			&\qquad \delta(q_1,\_) \rightarrow (q_{rejeita},\_,E);\\		
			&\qquad \delta(q_2,1) \rightarrow (q_1,1,D);
			\qquad \delta(q_2,0) \rightarrow (q_1,0,D);\\
			&\qquad \delta(q_2,\_) \rightarrow (q_3,\#,E);\\
			&\qquad \delta(q_3,1) \rightarrow (q_3,1,E);
			\quad \delta(q_3,0) \rightarrow (q_3,0,E);
			\quad \delta(q_3,\_) \rightarrow (q_4,\#,D);
			\quad \delta(q_3,\#) \rightarrow (q_8,\#,D);\\
			&\qquad \delta(q_4,1) \rightarrow (q_4,1,D);
			\qquad \delta(q_4,0) \rightarrow (q_4,0,D);
			\quad \delta(q_4,\#) \rightarrow (q_5,\#,E);\\
			&\qquad \delta(q_5,1) \rightarrow (q_6,\#,D);
			\quad \delta(q_5,0) \rightarrow (q_7,\#,D);
			\quad \delta(q_5,\#) \rightarrow (q_{11},\#,E);\\
			&\qquad \delta(q_6,\#) \rightarrow (q_3,1,E);\\
			&\qquad \delta(q_7,\#) \rightarrow (q_3,0,E);\\
			&\qquad \delta(q_8,1) \rightarrow (q_9,1,E);
			\qquad \delta(q_8,0) \rightarrow (q_{10},0,E);\\
			&\qquad \delta(q_9,\#) \rightarrow (q_4,1,D);\\
			&\qquad \delta(q_{10},\#) \rightarrow (q_4,0,D);\\
			&\qquad \delta(q_{11},1) \rightarrow (q_{11},1,E);
			\quad \delta(q_{11},0) \rightarrow (q_{11},0,E);
			\quad \delta(q_{11},\_) \rightarrow (q_{12},\_,D);
			\quad \delta(q_{11},X) \rightarrow (q_{12},X,D);\\
			&\qquad \delta(q_{12},1) \rightarrow (q_{13},X,D);
			\quad \delta(q_{12},0) \rightarrow (q_{14},X,D);
			\quad \delta(q_{12},\#) \rightarrow (q_{aceita},\#,E);\\
			&\qquad \delta(q_{13},1) \rightarrow (q_{13},1,D);
			\quad \delta(q_{13},0) \rightarrow (q_{13},0,D);
			\quad \delta(q_{13},\#) \rightarrow (q_{15},\#,D);\\
			&\qquad \delta(q_{14},1) \rightarrow (q_{14},1,D);
			\quad \delta(q_{14},0) \rightarrow (q_{14},0,D);
			\quad \delta(q_{14},\#) \rightarrow (q_{18},\#,D);\\
			&\qquad \delta(q_{15},\#) \rightarrow (q_{15},\#,D);
			\quad \delta(q_{15},X) \rightarrow (q_{15},X,D);
			\quad \delta(q_{15},0) \rightarrow (q_{rejeita},0,E);
			\quad \delta(q_{15},1) \rightarrow (q_{16},X,E);\\
			&\qquad \delta(q_{16},X) \rightarrow (q_{16},X,E);
			\quad \delta(q_{16},\#) \rightarrow (q_{17},\#,E);\\
			&\qquad \delta(q_{17},\#) \rightarrow (q_{17},\#,E);
			\quad \delta(q_{17},X) \rightarrow (q_{aceita},X,E);
			\quad \delta(q_{17},1) \rightarrow (q_{11},1,E);
			\quad \delta(q_{17},0) \rightarrow (q_{11},0,E);\\
			&\qquad \delta(q_{18},\#) \rightarrow (q_{18},\#,D);
			\quad \delta(q_{18},X) \rightarrow (q_{18},X,D);
			\quad \delta(q_{18},1) \rightarrow (q_{rejeita},1,E);
			\quad \delta(q_{18},0) \rightarrow (q_{16},X,E);
		\end{align*}
		E o diagrama de estados para $M$ Se encontra numa folha separada anexa.
		
		\item Informalmente, podemos definir a máquina seguindo os passos:
			\begin{enumerate}
    				\item Colocar o marcador na frente da entrada (início).
    				\item Percorrer a entrada inteira até o símbolo branco ($\_$), verificando se o tamanho lido é par. Caso contrário, rejeitar.
    				\item Colocar marcadores na frente e atrás da entrada.
    				\item Mover marcadores até que se encontrem no meio da entrada.
    				\item Comparar as palavras separadas pelos marcadores, símbolo a símbolo, marcando os símbolos lidos (e comparados). Se aparecerem símbolos diferentes na mesma posição duas palavras, rejeitar.
    				\item Aceitar.				
			\end{enumerate}
	\end{itemize}
	
\pagebreak

\problem*{6}
\question{Escreva um descrição informal e formal de uma máquina de Turing que decida as seguintes linguagens:}(\textit{Obs: Usou-se o software JFLAP para construção e validação das máquinas elaboradas.})
	\begin{itemize}
		\item $L = \{a^ib^jc^k: i \times j = k,\ i,j,k \geq 1\}$\\
		$Resp:$ Descrição informal:
		\begin{enumerate}
			\item Validar que a entrada tem pelo menos um de cada símbolo 'a', 'b' e 'c', necessariamente nessa ordem - para tanto, percorrer a entrada inteira da esquerda para a direita. Caso contrário, rejeitar.
			\item Retornar para o início, a esquerda da entrada, e marcamos o primeiro 'a' com um 'X'.
			\item Percorrer a entrada para a direita até encontrar o primeiro 'b'. Marcar esse 'b' com 'X' e prosseguir até o último 'c' da cadeia, se não houver nenhum 'c', rejeitar. Marcar esse 'c' com 'X'.
			\item Repetir o processo de marcar o primeiro 'b' a esquerda e o último 'c' a esquerda até não sobrar símbolos 'b' na fita.
			\item Voltar na fita para o primeiro 'a' a esquerda. Marcar o 'a' com 'X' e substituir cada 'X' a direita por 'b' até encontrar um 'c'. Voltar com a cabeça de leitura para o primeiro 'a' a direita.
			\item Repetir os passos a partir do item \textit{3.} até todos os 'a's, 'b's e 'c's estarem marcados com 'X'. Aceitar.
		\end{enumerate}
		\quad Descrição formal (Máquina $M_1$):
		\begin{align*}
			M_1 = & (Q,\Sigma, \Gamma, \delta, q_0, \square, \{aceita,rejeita\}):\\
			Q = & \{q_0,q_1,q_2,q_3,q_4,q_5,q_6,q_7,q_8,q_9,q_{10},
			q_{11},q_{12},q_{13},q_{14},q_{15},
			q_{16},q_{17},aceita,rejeita\};\\
			\Sigma = & \{a,b,c\};\\
			\Gamma = & \{a,b,c,X,\square\};\\
			\delta\ :\ & (Q - \{aceita,rejeita\}) \times \Gamma \rightarrow Q \times \Gamma \times \{L,R\}.
		\end{align*}
		As transições de $\delta$ são explicitadas no diagrama de estados a seguir para $M_1$, onde foram omitidas transições vazias ($\emptyset$) para maior clareza.\\
		\includegraphics[scale=.4]{"images/L7E6-1"}
		
\pagebreak
		
		\item $L = \{w \in \{0,1\}^*: w = w^R\}$ (a máquina deve ter duas fitas)\\
		$Resp:$ Descrição informal:
		\begin{enumerate}
			\item Com a entrada na primeira fita, percorrer as duas fitas para a direita, escrevendo os símbolos lidos na segunda fita.
			\item Retornar para o início da entrada com a primeira fita, mantendo a cabeça de leitura da segunda fita no final da "entrada" copiada.
			\item Se o comprimento da entrada for ímpar, percorrer a entrada para a direita novamente na primeira fita, simultaneamente, com a segunda fita para a esquerda. Validar, símbolo a símbolo, se todos os elementos da entrada conferem. Rejeitar se forem lidos símbolos diferentes nas cabeças de leitura.\\
			Se o comprimento da entrada for par faremos os mesmos passos do item anterior, mas com uma leitura "defasada" na primeira fita, isto é, a segunda fita estará uma leitura na frente da primeira. Dessa forma, validar na primeira lista o símbolo anteriormente lido na segunda e rejeitar se houver diferença. Aceitar com a primeira cabeça de leitura chegando no início da entrada.
		\end{enumerate}
		\quad Descrição formal (Máquina $M_2$):
		\begin{align*}
			M_1 = & (Q,\Sigma, \Gamma, \delta, q_0, \square, \{aceita,rejeita\}):\\
			Q = & \{q_0,q_1,q_2,q_3,q_4,q_5,q_6,aceita,rejeita\};\\
			\Sigma = & \{0,1\};\\
			\Gamma = & \{0,1,\square\};\\
			\delta\ :\ & (Q - \{aceita,rejeita\}) \times \Gamma \rightarrow Q \times \Gamma \times \{L,R\}.
		\end{align*}
		As transições de $\delta$ são explicitadas no diagrama de estados a seguir para $M_2$, onde foram omitidas transições vazias ($\emptyset$) para maior clareza.\\
		\includegraphics[scale=.4]{"images/L7E6-2"}

\pagebreak		
		
		\item $L = \{w  \{0,1\}^*: w = w^R\}$ (a máquina deve não determinística e ter duas fitas)\\
		$Resp:$ A descrição informal da máquina segue o mesmo raciocínio daquela do item anterior, porém fez-se o uso de $\lambda-$transições para facilitar a elaboração do diagrama de estados abaixo.\\
		\quad Descrição formal (Máquina $M_3$):
		\begin{align*}
			M_1 = & (Q,\Sigma, \Gamma, \delta, q_0, \square, \{aceita,rejeita\}):\\
			Q = & \{q_0,q_1,q_2,q_3,q_4,q_5,q_6,aceita,rejeita\};\\
			\Sigma = & \{0,1\};\\
			\Gamma = & \{0,1,\square\};\\
			\delta\ :\ & (Q - \{aceita,rejeita\}) \times \Gamma \rightarrow Q \times \Gamma \times \{L,R\}.
		\end{align*}
		As transições de $\delta$ são explicitadas no diagrama de estados a seguir para $M_3$, onde foram omitidas transições vazias ($\emptyset$) para maior clareza.\\
		\includegraphics[scale=.4]{"images/L7E6-3"}		
	\end{itemize}

\pagebreak

\problem*{7}
\question{Seja um $k$-AP um autômato com pilha que tem $k$ pilhas. Então, um $0$-AP é um
autômato finito não determinístico, e um $1$-AP é um autômato com pilha convencional. Já sabemos que $1$-APs são mais poderosos (reconhecem uma classe maior de linguagens) que $0$-APs.}
	\begin{enumerate}
		\item[a)] Mostre que 2-APs são mais poderosos que 1-APs.\\
		$Resp:$ Podemos pensar na linguagem $L = \{ww: w \in \{0,1\}^*\}$, que não é reconhecida por $1$-APs, mas com o seguinte $2$-AP $A = (Q,\Sigma,\Gamma,s,\delta,F)$:\\
		\begin{align*}
			Q & = \{s,q_1,q_2,q_3,q_f\};\\
			\Sigma & = \{1,0\};\\
			\Gamma & = \{1,0,\#\};\\
			F & = \{q_f\};\\
			\delta : \\
			&\delta(s, \lambda, \lambda, \lambda) \rightarrow \{(q_1,\#,\#)\};\\
			&\delta(q_1, 1, \lambda, \lambda) \rightarrow \{(q_1,1,\lambda)\};\\
			&\delta(q_1, 0, \lambda, \lambda) \rightarrow \{(q_1,0,\lambda)\};\\
			&\delta(q_1, \lambda, \lambda, \lambda) \rightarrow \{(q_2,\lambda,\lambda)\};\\
			&\delta(q_2, \lambda, 1, \lambda) \rightarrow \{(q_2,\lambda,1)\};\\
			&\delta(q_2, \lambda, 0, \lambda) \rightarrow \{(q_2,\lambda,0)\};\\
			&\delta(q_2, \lambda, \#,\lambda) \rightarrow \{(q_3,\lambda,\lambda)\};\\
			&\delta(q_3, 1, \lambda, 1) \rightarrow \{(q_3,\lambda,\lambda)\};\\
			&\delta(q_3, 0, \lambda, 0) \rightarrow \{(q_3,\lambda,\lambda)\};\\
			&\delta(q_3, \lambda, \lambda, \#) \rightarrow \{(q_f,\lambda,\lambda)\};\\
			&(demais\ \delta 's = \emptyset)
		\end{align*}
		Como $A$ reconhece uma linguagem que $1$-APs não reconhecem, ele pertence a uma classe de autômatos mais poderosos.				
		\item[b)] Mostre que 3-APs não são mais poderosos que 2-APs.(Sugestão: Simule uma fita de uma máquina de Turing com duas pilhas.
	\end{enumerate}

\problem*{8}
\question{Mostre que a classe de linguagens decidíveis é fechada sob as operações de:
união, concatenação, estrela e intersecção.}
	$Resp:$ Dadas as maquinas de Turing $M_1$ e $M_2$, que reconhecem as linguagens $L_1$ e $L_2$, conseguimos construir uma terceira máquina $M_3$ que funciona da seguinte forma:
	\begin{itemize}
		\item Decide $L_1 \cup L_2$: para uma entrada $w \in \Sigma^*$ executar as máquinas $M_1$ e $M_2$ para $w$ e aceitamos a entrada, se e somente se, qualquer uma das duas máquinas aceitar.
		\item Decide $L_1 \cap L_2$: para uma entrada $w \in \Sigma^*$ executar as máquinas $M_1$ e $M_2$ para $w$ e aceitamos a entrada, se e somente se, as duas máquinas aceitarem.
		\item Decide $\overline{L_1}$: para uma entrada $w \in \Sigma^*$ executar a máquina $M_1$ para $w$ e aceitar a entrada, se $M_1$ rejeitar, e rejeitar se $M_1$ aceitar.
		\item Decide $L_1^*$: para uma entrada $w \in \Sigma^*$, se $w = \lambda$, aceitar. Caso contrário, para as $2^{|w|-1}$ formas de dividir $w$ em fatores $w_1w_2...w_i...w_k, w_i \neq \lambda$, executar a máquina $M_1$ para cada fator de $w$ e aceitar a entrada, se e somente se, $M_1$ aceitar todos os fatores. Se $M_1$ rejeitar qualquer fator como entrada, rejeitar.
		\item Decide $L_1.L_2$: para uma entrada $w \in \Sigma^*$, para cada $|w| + 1$ formas de obter $w = xy$, executar a máquina $M_1$ para $x$ e $M_2$ para $y$, aceitando a entrada, se e somente se, ambas $M_1$ e $M_2$ aceitarem. Caso contrário, rejeitamos.
	\end{itemize}

\problem*{9}
\question{Mostre que a classe de linguagens Turing-reconhecíveis é fechada sob as operações
de:  união, concatenação, estrela e intersecção.}	
	$Resp:$ Dadas as maquinas de Turing $M_1$ e $M_2$, que reconhecem as linguagens $L_1$ e $L_2$, conseguimos construir uma terceira máquina $M_3$ que funciona da seguinte forma:
	\begin{itemize}
		\item Reconhece $L_1 \cup L_2$: para uma entrada $w \in \Sigma^*$ executar as máquinas $M_1$ e $M_2$ para $w$ em paralelo e aceitar a entrada, se e somente se, qualquer uma das duas máquinas aceitar.
		\item Reconhece $L_1 \cap L_2$: para uma entrada $w \in \Sigma^*$ executar as máquinas $M_1$ e $M_2$ para $w$ em paralelo e aceitar a entrada, se e somente se, as duas máquinas aceitarem.
		\item Reconhece $L_1^*$: para uma entrada $w \in \Sigma^*$, se $w = \lambda$, aceitar. Caso contrário, para as $2^{|w|-1}$ formas de dividir $w$ em fatores $w_1w_2...w_i...w_k, w_i \neq \lambda$, executar em paralelo a máquina $M_1$ para cada fator de $w$ e aceitar a entrada, se e somente se, $M_1$ aceitar todos os fatores. Se $M_1$ rejeitar qualquer fator como entrada, rejeitar.
		\item Reconhece $L_1.L_2$: para uma entrada $w \in \Sigma^*$, para cada $|w| + 1$ formas de obter $w = xy$, executar em paralelo a máquina $M_1$ para $x$ e $M_2$ para $y$, aceitando a entrada, se e somente se, ambas $M_1$ e $M_2$ aceitarem. Caso contrário, rejeitamos.
	\end{itemize}
\end{document}
