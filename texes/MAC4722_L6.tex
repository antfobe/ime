% to change the appearance of the header, questions, problems or subproblems, see the homework.cls file or
% override the \Problem, \Subproblem, \question or \printtitle commands.

% The hidequestions option hides the questions. Remove it to print the questions in the text.

\title{MAC4722 - Lista 6}

\documentclass{homework}

\usepackage[utf8]{inputenc}
\usepackage{amsthm}
\usepackage{mathtools}

\usepackage{tikz}
\usetikzlibrary{automata,positioning}

\usepackage{graphicx}
\graphicspath{ {images/} }

\newtheorem*{theorem}{Forma Normal de Chomsky}

% Set up your name, the course name and the homework set number.
\homeworksetup{
    username={Jean Fobe, N$^o$USP 7630573},
    course={Linguagens, Autômatos e Computabilidade - MAC4722},
    setnumber=6}
\begin{document}% this also prints the header.
\pagestyle{fancy}
\fancyfoot[L]{Jean Fobe, 7630573}

% use starred problems or subproblems to apply manual numbering.
\problem*{2}
\question{Escreva uma gramática livre de contexto para cada uma das linguagens a seguir. Explique cada uma das regras de substituição de sua gramática para justificar porque a linguagem gerada pela sua gramática é a correspondente linguagem do enunciado.}
	\begin{itemize}
		\item[(b)] O complemento da linguagem $\{a^n b^n:\ n \geq 0\}$\\
		$Resp:$ Antes de escrever uma gramática vamos antes explicitar o complemento da linguagem acima nos seguintes casos (Admitimos aqui que $\Sigma = \{a,b\}$):
		\begin{enumerate}
			\item $\{a^n b^m:\ n,m \geq 0\ e\ n \neq m\}$;
			\item $\{b x:\ x \in \{a,b\}^*\}$;
			\item $\{a^n b^n y:\ n \geq 0,\ y \in \{a,b\}^+\}$;
		\end{enumerate}		 Existe uma gramática $G = (V, \Sigma, R, S)$ que gera o complemento de $\{a^n b^n:\ n \geq 0\}$, onde: % ($https://math.stackexchange.com/questions/550756/how-to-create-a-grammar-for-complement-of-anbn$)
		\begin{align*}
			V &= \{S\};\\
			\Sigma &= \{a,b\};\\
			R &= \{S \rightarrow A\ |\ B\ |\ C,\\
			(1^o\ caso)\ A & \rightarrow Ab\ |\ aA\ |\ F,\\
			(2^o\ caso)\ B & \rightarrow bF\ |\ b,\\
			(3^o\ caso)\ C & \rightarrow aCF\ |\ F,\\
			F & \rightarrow FF\ |\ a\ |\ b\};
		\end{align*}
		\item[(c)] $\{a^m b^n c^p d^q:\ m,n,p,q \geq 0\ e\ m + n = p + q\}$\\
		$Resp:$ Extraímos da descrição acima que $|a^m b^n| = |c^p d^q|$, então uma gramática $G = (V, \Sigma, R, S)$ para gerar essa linguagem pode ser dada da seguinte forma: % $(https://cs.stackexchange.com/questions/83150/is-this-language-context-free-l-ambncpdq-mn-pq)$
		\begin{align*}
			V &= \{A, B, C\};\\
			\Sigma &= \{a,b,c,d\};\\
			R &= \{S \rightarrow ABC,\\
				&\qquad	A \rightarrow a\ |\ \lambda, \\
				&\qquad	B \rightarrow bBc\ |\ aABCd\ |\ \lambda, \\
				&\qquad	C \rightarrow d\ |\ \lambda\};
		\end{align*}
	\end{itemize}
\pagebreak

\problem*{3}
\question{Mostre que se G é uma gramática livre de contexto na Forma Normal de Chomsky então, para cada palavra não vazia $w \in L(G)$ são necessários exatamente $2|w| - 1$ passos em qualquer derivação de w.}
	\begin{theorem}
		(Sipser p. 107, definição 2.8) Uma gramática livre de contexto está na \textbf{forma normal de Chomsky} se todas as suas regras são dadas na forma:
		\begin{align*}
			A & \rightarrow BC; \\
			A & \rightarrow a;
		\end{align*}
		Onde $a$ is any terminal e $A,B,C$ são variáveis quaisqueres, com $B$ ou $C$ não sendo variável de início. Adicionalmente, é permitida a regra $S \rightarrow \lambda$, em que $S$ é a variável inicial.  
	\end{theorem}
	$Resp:$ Seja $|w| = n$. Podemos derivar a palavra $w$ a partir de uma gramática $G$ na \textbf{forma normal de Chomsky}. A derivação inicia pela variável $S$, com $|S| = 1$.\\
	Usando $n-1$ regras da forma $A \rightarrow BC\ (variavel \rightarrow variavel\ variavel)$, construímos uma palavra formada por apenas variáveis de tamanho $n$.\\
	Podemos, então, aplicar a regra do tipo $A \rightarrow a\ (variavel \rightarrow simbolo)$ na palavra de tamanho $n$, realizando mais $n$ derivações e chegando na palavra $w$. No total aplicamos $(n - 1) + n = 2n -1$ passos de derivação para chegar em $w$. Em expressões:
		\begin{align*}
			& |w| = n; \\			
			& S \xRightarrow[\text{G}]{} V_1V_2\ \xRightarrow[\text{G}]{n-2} V_1V_2...V_n,\  (V_1,V_2,...,V_n\ todas\ variaveis\ de\ G); \\
			& V_1V_2...V_n \xRightarrow[\text{G}]{n} w;
		\end{align*}

\problem*{4}
\question{Considere a linguagem $L = \{a^i b^j c^k: i,j,k \geq 0\ e\ (i = 0\ ou\ j = k)\}.$}
	\begin{itemize}
		\item[(a)] Prove que L não é regular.\\
		$Resp:$ Vamos provar por contradição que $L$ não é regular.\\
			Se $L$ é regular, sabendo que o conjunto das linguagens regulares é fechado para operação de união, podemos descrever $L$ como uma união de linguagens regulares: 
			\[L = L_1 \cup L_2,\ L_1 = \{b^j c^k:j,k \geq 0\}\ e\ L_2 = \{a^i b^j c^j:i,j \geq 0\}\] 
			Podemos também fazer uma intersecção $L \cap L_{bc}$, com $L_{abc} = \{ab^x c^y:x,y \geq 0\}$ uma linguagem regular, e teremos outra linguagem regular como resultado, dado que o conjunto de linguagens regulares é fechado para essa operação:
			\begin{align*}
				L_{aux} &= L\ \cap\ L_{abc} = \{ab^j c^j:j \geq 0\}\ e\\
				L_{aux} &= \{a\}\ .\ L_{nr},\ L_{nr} = \{b^j c^j:j \geq 0\}
			\end{align*}
			Como sabemos que $L_{nr}$ não é regular (é uma linguagem da forma $\{a^nb^n:n \geq 0\}$), chegamos em uma contradição, porque partindo de operações entre linguagens regulares obtivemos $L_{nr}$ como resultado. Então, concluímos que $L$ não é regular.
	\end{itemize}
	\begin{itemize}
		\item[(b)] Forneça um inteiro positivo p e mostre que a linguagem L satisfaz
as condições do Lema do Bombeamento para esse p.\\
		$Resp:$ Seja $p$ o inteiro fornecido pelo \textbf{Lema do Bombeamento} para $L$ e a palavra $w = a^pb^jc^j$. Tomando $p = 1,$ para $j \geq 0$. Com $x = a^0, y = a^p, p \geq |xy| \geq |y| \geq 1\ e\ z = b^j c^j$, considere todas as palavras que satisfazem ($i \geq 0$):
			\[xy^iz = a^0(a^{p})^i b^j c^j = a^{pi} b^j c^j\]
		Observamos que para todo inteiro $i,j \geq 0$ o \textbf{LB} é satisfeito. Então, para um inteiro $p = 1$, teremos um conjunto de palavras $W = a^{i} b^j c^j$, todas elas pertencendo à $L$.
	\end{itemize}
\end{document}
