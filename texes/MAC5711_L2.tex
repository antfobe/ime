% to change the appearance of the header, questions, problems or subproblems, see the homework.cls file or
% override the \Problem, \Subproblem, \question or \printtitle commands.

% The hidequestions option hides the questions. Remove it to print the questions in the text.

\title{MAC5711- Lista 2}

\documentclass{homework}

\usepackage[utf8]{inputenc}
\usepackage{amsthm}
\usepackage{tikz}
\usetikzlibrary{automata,positioning}

\usepackage{graphicx}
\graphicspath{ {images/} }

\usepackage{mathtools}
\DeclarePairedDelimiter\ceil{\lceil}{\rceil}
\DeclarePairedDelimiter\floor{\lfloor}{\rfloor}

\usepackage[noend]{algpseudocode}

% Set up your name, the course name and the homework set number.
\homeworksetup{
    username={Jean Fobe, N$^o$USP 7630573},
    course={Análise de Algorítmos - MAC5711},
    setnumber=1}
\begin{document}% this also prints the header.
\pagestyle{fancy}
\fancyfoot[L]{Jean Fobe, 7630573}

% use starred problems or subproblems to apply manual numbering.
\problem*{1}
\question{Resolva as recorrências abaixo.}
	\begin{enumerate}
		\item[(b)] $T(n) = 8T(\floor{n/2})+\Theta(n^2)$
		\item[$Resp:$] Vamos resolver a recorrência por expansão em $k, k = lg\ n$. Observando que $T(1) = 1$ e simplificando $T(n) = 8T(n/2)+\Theta(n^2)$, para $n \geq 2$ potência de 2. Por expansão:
				\begin{align*}
					& T(n) = 8^kT\left(1\right) + \sum_{i=0}^{k-1} \left(\frac{8}{2}\right)^i.\ n^2\\
					& T(n) = 8^{lgn}T\left(1\right) + \sum_{i=0}^{lgn-1} \left(\frac{8}{2}\right)^i.\ n^2\\	
					& T(n) = 8^{lgn} + \left( \frac{\left( \frac{8}{2} \right)^{lgn} - 1}{\frac{8}{2} - 1} \right).\ n^2\\
					& T(n) = n^4 + \frac{n^4}{3} .\ n^2\\
					& T(n) \leq n^4 + \frac{n^6}{3} = \mathsf{O}(n^6)
				\end{align*}
		Temos, então, que a recorrência resulta num consumo de tempo O($n^6$).
	\end{enumerate}	
	
\pagebreak	
		
\problem*{3}
\question{Seja $X[1 .. n]$ um vetor de inteiros e $i$ e $j$ dois índices distintos de $X$, ou seja, $i$ e $j$ são inteiros entre $1$ e $n$. Dizemos
  que o par $(i,j)$ é uma \emph{inversão} de $X$ se $i<j$ e $X[i]>X[j]$. Escreva um algoritmo O$(n \lg n)$ que devolva o número de inversões em um
  vetor $X$, onde $n$ é o número de elementos em $X$.}
	\begin{enumerate}
		\item[$Resp:$] Segue o algoritmo (admitiu-se que o menor tamanho de $X$ é $n = 2$ e $Y[p..n]$ é vetor auxiliar, com $p \leq n$):\\
			$\mathsf{InvertedCount}(X,p,n)$
			\begin{algorithmic}[1]
				\State $count\gets 0$
				\If {$n > 1$}
					\State $q\gets \floor{\frac{n+1}{2}}$
					\State $count\gets \mathsf{InvertedCount}(X,p,q)$
					\State $count\gets count + \mathsf{InvertedCount}(X,q+1,n)$
					\For {$i\gets p$ to $q$}
						\State $Y[i]\gets X[i]$
					\EndFor
					\For {$j\gets q+1$ to $n$}
						\State $Y[n+q+1-j]\gets X[j]$
					\EndFor
					\State $i\gets p$
					\State $j\gets n$
					\For {$k\gets p$ to $n$}
						\If {$Y[i] < X[j]$}
							\State $X[k]\gets Y[i]$
							\State $i\gets i + 1$
						\Else
							\State $X[k]\gets Y[j]$
							\State $j\gets j - 1$
							\State $count\gets count + 1$							
						\EndIf
					\EndFor			
				\EndIf
				\State return count
			\end{algorithmic}
		Observando o algoritmos acima, temos uma similaridade quase completa com o $\mathsf{MergeSort}(A,p,r)$ visto em sala, as únicas mudanças sendo a inclusão no início do algoritmo da variável $count$, o seu incremento quando $Y[i] > X[j]$ (linha 19) e os retornos de função (algoritmo).\\ 
		Sabendo que $\mathsf{MergeSort}$ é O$(n \lg n)$, podemos ver que $\mathsf{InvertedCount}$ tem o mesmo consumo de tempo.
	\end{enumerate}
\end{document}
