%%%%%%%%%%%%%%%%%%%%%%%%%%%%%%%%%%%%%%%%%
% Short Sectioned Assignment
% LaTeX Template
% Version 1.0 (5/5/12)
%
% This template has been downloaded from:
% http://www.LaTeXTemplates.com
%
% Original author:
% Frits Wenneker (http://www.howtotex.com)
%
% License:
% CC BY-NC-SA 3.0 (http://creativecommons.org/licenses/by-nc-sa/3.0/)
%
%%%%%%%%%%%%%%%%%%%%%%%%%%%%%%%%%%%%%%%%%

%----------------------------------------------------------------------------------------
%	PACKAGES AND OTHER DOCUMENT CONFIGURATIONS
%----------------------------------------------------------------------------------------

\documentclass[paper=a4, fontsize=11pt]{scrartcl} % A4 paper and 11pt font size

\usepackage[T1]{fontenc} % Use 8-bit encoding that has 256 glyphs
\usepackage{fourier} % Use the Adobe Utopia font for the document - comment this line to return to the LaTeX default
\usepackage[english]{babel} % English language/hyphenation
\usepackage{amsmath,amsfonts,amsthm} % Math packages

\usepackage{lipsum} % Used for inserting dummy 'Lorem ipsum' text into the template

\usepackage{sectsty} % Allows customizing section commands
\allsectionsfont{\normalfont\scshape} % Make all sections centered, the default font and small caps

\usepackage{fancyhdr} % Custom headers and footers
\pagestyle{fancyplain} % Makes all pages in the document conform to the custom headers and footers
\fancyhead{} % No page header - if you want one, create it in the same way as the footers below
\fancyfoot[L]{} % Empty left footer
\fancyfoot[C]{} % Empty center footer
\fancyfoot[R]{\thepage} % Page numbering for right footer
\renewcommand{\headrulewidth}{0pt} % Remove header underlines
\renewcommand{\footrulewidth}{0pt} % Remove footer underlines
\setlength{\headheight}{13.6pt} % Customize the height of the header

\numberwithin{equation}{section} % Number equations within sections (i.e. 1.1, 1.2, 2.1, 2.2 instead of 1, 2, 3, 4)
\numberwithin{figure}{section} % Number figures within sections (i.e. 1.1, 1.2, 2.1, 2.2 instead of 1, 2, 3, 4)
\numberwithin{table}{section} % Number tables within sections (i.e. 1.1, 1.2, 2.1, 2.2 instead of 1, 2, 3, 4)

\setlength\parindent{0pt} % Removes all indentation from paragraphs - comment this line for an assignment with lots of text

%----------------------------------------------------------------------------------------
%	TITLE SECTION
%----------------------------------------------------------------------------------------

\newcommand{\horrule}[1]{\rule{\linewidth}{#1}} % Create horizontal rule command with 1 argument of height

\title{	
\normalfont \normalsize 
\textsc{USP, Instituto de Matemática e Estatística} \\ [25pt] % Your university, school and/or department name(s)
\horrule{0.5pt} \\[0.4cm] % Thin top horizontal rule
\huge MAC6955: Estado da arte em testes de software \\ % The assignment title
\normalfont \normalsize 
Professor Doutor Mauricio Aniche \\ [25pt]
\horrule{2pt} \\[0.5cm] % Thick bottom horizontal rule
}

\author{Jean Luc A. O. Fobe, \\
        NºUSP: 7630573} % Your name

\date{13/12/2018} % Today's date or a custom date

\begin{document}

\maketitle % Print the title

\section{Perguntas}
\subsection{Explique o problema do oráculo. (\textit{1 ponto})}
\begin{itemize}
    \item[Resp:] O problema do oráculo pode ser resumido em determinar o \textit{output} ou saída de um programa de forma correta para um dado \textit{input} ou entrada, que pode ser dada como sequencia de estados ou até mesmo outros programas.\\
    Cem Kaner relata que um dos problemas chaves que aparecem com oráculos é que eles atendem a uma parte pequena de \textit{inputs} e \textit{outputs} que realmente estão relacionados com qualquer teste (\textit{A Course in Black Box Software Testing}, 2004). O programador de testes pode modificar valores das variáveis do programa, mas a execução em si também depende do sistema onde o programa está rodando e isso afeta o resultado dos testes. Logo, a avaliação dos resultados dos testes em função dos \textit{inputs} tem base em dados incompletos que podem torna-la inválida.
\end{itemize}

\subsection{Explique a diferença entre line coverage, branch coverage, condition coverage, e path coverage. Fique à vontade para fazer uso de exemplos. \textit{(2 pontos)}}
    \begin{itemize}
        \item[Resp:] Myers resume \textit{branch covarage} e \textit{condition coverage} (\textit{The Art of Software Testing, Chapter 4}. Glenford J. Myers (2004)):
        \begin{itemize}
            \item \textit{Branch Coverage}: Métrica de testes feitos sobre toda ramificação (\textit{branch}, DD-path) de toda estrutura de controle (\textit{if, case statements}). Por exemplo, dada uma declaração \textit{if}, ambos os valores para o ramo "Verdadeiro" e o "Falso" devem ser testados.
            \item \textit{Condition Coverage}: Cada \textit{eval} de todas as sub expressões booleanas deve ser testado para os valores "Verdadeiro" e "Falso". (Não somente condições de estruturas de controle).
        \end{itemize}
         Das transparências passadas durante o curso (\textit{2018 test automation day}) e do material de Steve Cornett (\textit{Code Coverage Analysis}) podemos ainda diferenciar \textit{line coverage} e \textit{path coverage}:
         \begin{itemize}
             \item \textit{Line Coverage}: Informa para dado teste quais linhas de código foram percorridas.
             \item \textit{Path Coverage}: Considera se cada caminho possível entre estados do programa foi percorrido, onde um caminho entre dois estados é uma sequencia de transições que levam a execução do programa de um estado para o outro. Esse tipo de métrica é exponencial em número de condições. Se temos um trecho de código que é uma transição de um estado para outros com 10 condições \textit{if-else} testaríamos um total de $2^{10} = 1024$ `subcaminhos' somente nessa parte do programa.
         \end{itemize}
         Considere o trecho de código C abaixo:
         \begin{align*}
             & int\ fxor(int\ x,\ int\ y)\{\\
             & \qquad int\ z=0;\\
             & \qquad if((x>0)\ \&\&\ (y>0))\\
             & \qquad \qquad z=x \wedge y;\\
             & \qquad return\ flip(z);\\
             & \}\\
             & int\ flip(int\ z)\{\\
             &\qquad if(z>0)\\
             &\qquad \qquad return\ z \wedge\ 0xFF;\\
             &\qquad return\ -1;\\
             &\}
         \end{align*}
         Um teste com valores $x = 1,\ y = 1$, terá \textit{line coverage} de $91\%$ ($9/10$ linhas percorridas), enquanto que para valores $x = 0,\ y = 1$, \textit{line coverage} será de $~82\%$ ($9/11$ linhas).\\
          $x = 0,\ y = 1$, terá \textit{path coverage} de $25\%$, pois apenas um caminho dentre os quatro possíveis foi percorrido. O mesmo ocorre para \textit{branch coverage} uma vez que só $25\%$ das ramificações é testado. Esse \textit{input} também faz com que o trecho tenha \textit{condition coverage} de $12.5\%$, já que somente um oitavo do resultado de expressões booleanas foi testado.        
    \end{itemize}

\subsection{Em search-based software testing, a estratégia mais popular é desenhar um algoritmo genético capaz de gerar testes para um dado programa. O grande segredo que faz o algoritmo genético ser bem sucedido é a sua fitness function. Explique, em detalhes, a fitness function discutida em aula. \textit{(3 pontos)}}
    \begin{itemize}
        \item[Resp:] Resumidamente, uma \textit{fitness function} é uma função que avalia o quão perto da solução ótima está a resposta dada para resolver um dado problema - determina o quanto a solução se `encaixa' no problema como resposta.\\
        A \textit{fitness function} discutida em aula avalia para cada \textit{query} SQL o nível da query (admitindo \textit{nesting}) e a distancia dada pela quantidade de expressões emparelhadas que resultam no \textit{output} desejado, ou \textit{target}. Essa abordagem permite que um número muito menor de execuções sejam realizadas para testar os casos desejados, uma vez que o algoritmo tenta convergir o mais rápido o possível para os alvos de teste. Em outras palavras, ao invés de testar cada condição, ou cada expressão lógica, ou ainda cada caminho da aplicação, a \textit{target function} permite que o algoritmo genético se adapte de acordo com as partes de código que já foram cobertas.\\
        Para o exemplo dado em sala, com dois níveis de \textit{queries} e três condições, para chegarmos na avaliação de falso para a terceira condição da primeira \textit{query}, queremos um teste que encaixe em:
        \begin{align*}
           & fitness(t)=step\_level(t)+branch\_distance(t),\\
           & step\_level(t)=1;\ branch\_distance(t)=3;
        \end{align*}         
    \end{itemize}

\subsection{Defina e explique a diferença entre mutation testing e fuzzing testing. \textit{(2 pontos)}}
	\begin{itemize}
		\item[Resp:] DeMillo define \textit{mutation testing} como a criação de um conjunto de programas mutantes do programa original sendo testado. Cada mutação difere do original por uma mutação e cada mutação é uma única mudança sintática feita em uma declaração ou condição do programa (Richard A. DeMillo, Richard J. Lipton, and Fred G. Sayward. \textit{Hints on test data selection: Help for the practicing programmer}. IEEE Computer, 11(4):34-41. April 1978).\\
		OWASP (\textit{Open Web Application Security Project}) define \textit{fuzzing} como uma técnica \textit{caixa preta} de teste de software, que consiste basicamente em encontrar defeitos de implementação usando expressões mal estruturadas ou parcialmente corretas para injeção de dados de uma forma automática.
		A diferença mais clara entre essas duas técnicas é em que parte do programa cada uma atua, \textit{fuzzing} atua nos \textit{inputs} do programa, tentado produzir entradas que quebrem a execução do programa, e \textit{mutation testing} testa a lógica do programa, mostrando se alguma substituição de operadores ou variáveis tem efeito inesperado.
	\end{itemize}

\pagebreak

\subsection{Apesar de interessantes e eficientes, usar essas técnicas em projetos de software reais ainda é desafiador. Discuta 3 desses desafios. \textit{(2 pontos)}}
	\begin{itemize}
		\item[Resp:] Três principais problemas discutidos em aula foram:
		\begin{enumerate}
			\item Automação de testes: Principalmente na parte de \textit{design} do software ainda faltam técnicas confiáveis que permitam testes com boa cobertura dos programas produzidos. No final do curso o professor discutiu algumas maneiras interessantes de resolver esse problema, tanto pelo bom aproveitamento dos \textit{logs} da aplicação quanto pela implementação do algoritmo genético para testes de SQL.
			\item A ordem de testes ou quais testes executar primeiro: Tendo como objetivo achar \textit{bugs} no software, não há forma genérica de decidir a ordem dos testes e esse problema atrasa o \textit{deploy} da aplicação e acaba acarretando em mais custos para o processo de desenvolvimento.
			\item Testar a aplicação por inteiro: Além de ser uma atividade custosa de tempo e recursos, não é nada trivial ter uma cobertura de testes completa sobre o software. Surgem problemas como o oráculo de teste e tempo exponencial no tamanho do programa.
		\end{enumerate}
	\end{itemize}

\section{Avaliação pessoal}

Em uma escala de 0 a 10, onde 10 significa que você está apto a implementar todas as técnicas que discutimos, bem como talvez até trabalhar em um artigo científico que visa expandir o conhecimento atual, como você se avalia?


\subsection{Indique a sua avaliação pessoal e escreva um parágrafo explicando o porquê.}
\begin{itemize}
    \item[-] Atribuo uma nota de 7.5.\\
    Gostei muito do curso e me interesso por essa área de conhecimento, mas não vejo formas diretas de aplicar o que foi aprendido. Busquei participar por meio de perguntas e discussões e isso com certeza me ajudou a entender melhor o tema. Talvez até consiga trabalhar em artigos científicos envolvendo uso de \textit{logs} para descoberta de \textit{bugs} ou inconsistências em aplicações.
\end{itemize}

\subsection{Se você se deu uma nota maior ou igual à 8, justifique-se em detalhes.}

\begin{itemize}
    \item[-] A nota não foi maior que 8.
\end{itemize}

\end{document}