% to change the appearance of the header, questions, problems or subproblems, see the homework.cls file or
% override the \Problem, \Subproblem, \question or \printtitle commands.

% The hidequestions option hides the questions. Remove it to print the questions in the text.

\title{MAC4722 - Lista 5}

\documentclass{homework}

\usepackage[utf8]{inputenc}
\usepackage{amsthm}
\usepackage{tikz}
\usetikzlibrary{automata,positioning}

\usepackage{graphicx}
\graphicspath{ {images/} }

\newtheorem*{theorem}{Lema do Bombeamento}

% Set up your name, the course name and the homework set number.
\homeworksetup{
    username={Jean Fobe, N$^o$USP 7630573},
    course={Linguagens, Autômatos e Computabilidade - MAC4722},
    setnumber=5}
\begin{document}% this also prints the header.
\pagestyle{fancy}
\fancyfoot[L]{Jean Fobe, 7630573}

% use starred problems or subproblems to apply manual numbering.
\problem*{1}
\question{Prove que as seguintes linguagens não são regulares. Pode utilizar o Lema do Bombeamento e as propriedades de fechamento da classe de linguagens regulares (vistas em aula ou nas listas de exercícios).}
	\begin{theorem}
		Seja L uma linguagem regular. Então existe um inteiro $p \geq 1$ (chamado 		comprimento de bombeamento) tal que cada cadeia w de L com comprimento maior ou igual a p pode ser escrita como w = xyz (i.e. w pode ser dividida em três subcadeias) satisfazendo as seguintes condições:
		\begin{enumerate}
			\item $|y| \geq 1;$
			\item $|xy| \leq p;$
			\item $xy^iz \in L, \forall i \geq 0.$
		\end{enumerate}
	\end{theorem}
	\begin{enumerate}
		\item[(b)] $L = \{w \in \{a,b\}^*: w$ não é um palíndromo$\}$\\
			$Resp:$ Vamos provar por contradição que $L$ não é regular.\\
			Se $L$ é regular, sabendo que o conjunto das linguagens regulares é fechado para operação de complemento, temos que $\overline{L}$ é regular também. Sendo $\overline{L}$ linguagem de todos os palíndromos em $\{a,b\}^*$, \[\overline{L} = \{k \in \{a,b\}^*:k=k^R\}\]
			Seja $p$ o inteiro fornecido pelo \textbf{Lema do Bombeamento} para $\overline{L}$ e a palavra $w \in \overline{L}, w = a^pbba^p$. Com $x = a^r, y = a^s, p \geq |xy| \geq |y| \geq 1\ e\ z = a^{p-r-s}bba^p$, considere a palavra ($i = 3$):
			\[t = xy^3z = a^r(a^s)^3a^{p-r-s}bba^p = a^{p+2s}bba^p\]
			Como $t^R = a^pbba^{p+2s}$ e $t^R \neq t,\ t \notin \overline{L}$. Isso contradiz a condição \textit{3} do \textbf{LB}.\\
			Então, sabendo que o complemento de $L\ (\overline{L})$ não é regular, colocamos que $L$ não é regular também.
			
		\item[(c)] $L = \{wtw: w,t \in \{a,b\}^+\}$\\
			$Resp:$ Vamos provar por contradição que $L$ não é regular.\\
			Vamos supor que $L$ é regular, seja $p$ o inteiro fornecido pelo \textbf{Lema do Bombeamento} para $L$ e a palavra $v = ab^paab^p$ (note que $t = a$ e $|t| = 1$, nesse caso). Com $x = ab^r, y = b^{s}, \\p \geq |xy| \geq |y| \geq 1\ e\ z = b^{p-r-s}aab^p$, considere a palavra ($i = 3$):
			\[j = xy^3z = ab^r(b^{s})^3b^{p-r-s}aab^p = ab^{p+2s}aab^p\]
			Como $|j| = 2p+2s+3$, o prefixo e o sufixo de $j$ de comprimento $\tfrac{|j|-1}{2}$ são $j_1=ab^{p+s}$ e $j_2=b^{s-1}aab^p$, respectivamente. Tendo $j_1 \neq j_2,\ j = j_1tj_2 \notin L$. Isso contradiz a condição \textit{3} do \textbf{LB}.\\
			Então, $L$ não é uma linguagem regular.
	\end{enumerate}
	
\pagebreak

\problem*{2}
\question{Considere as linguagens:
	\begin{align*}
		A &= \{b^kx: x \in \{a,b\}^*\ e\ |x|_b \geq k \geq 1\}\ e\\
		B &= \{b^kx: x \in \{a,b\}^*\ e\ |x|_b \leq k,\ k \geq 1\}.	
	\end{align*}	 
}
	\begin{enumerate}
		\item[(b)] Prove que $B$ não é linguagem regular.\\
		$Resp:$ Vamos provar por contradição que $B$ não é regular.\\
			Supondo que $B$ é regular, seja $p$ o inteiro fornecido pelo \textbf{Lema do Bombeamento} para $B$ e a palavra $w = b^pab^pa$. Com $x = b^r, y = b^{s}, p \geq |xy| \geq |y| \geq 1\ e\ z = b^{p-r-s}ab^pa$, considere a palavra ($i = 0$):
			\[t = xy^0z = b^r(b^s)^0b^{p-r-s}ab^pa = b^{p-s}ab^pa\]
			Como $|t|=2p-s+2$ e o prefixo com somente `b's da palavra $w,\ w_p$ é tal que $|w_p|_b = p - s$ e o resto da palavra, o sufixo $w_s$, fica com $|w_s|_b = p$. Sendo $|w_s|_b > |w_p|_b$, $w \notin B$, contradizendo a condição \textit{3} do \textbf{LB}.\\
			Portanto $B$ não é regular.
	\end{enumerate}
	
\problem*{3}
\question{Considere a linguagem $L = \{a^i b^j c^k: i,j,k \geq 0\ e\ (i = 0\ ou\ j = k)\}.$}
	\begin{itemize}
		\item[(a)] Prove que L não é regular.\\
		$Resp:$ Vamos provar por contradição que $L$ não é regular.\\
			Se $L$ é regular, sabendo que o conjunto das linguagens regulares é fechado para operação de união, podemos descrever $L$ como uma união de linguagens regulares: 
			\[L = L_1 \cup L_2,\ L_1 = \{b^j c^k:j,k \geq 0\}\ e\ L_2 = \{a^i b^j c^j:i,j \geq 0\}\] 
			Podemos também fazer uma intersecção $L \cap L_{bc}$, com $L_{abc} = \{ab^x c^y:x,y \geq 0\}$ uma linguagem regular, e teremos outra linguagem regular como resultado, dado que o conjunto de linguagens regulares é fechado para essa operação:
			\begin{align*}
				L_{aux} &= L\ \cap\ L_{abc} = \{ab^j c^j:j \geq 0\}\ e\\
				L_{aux} &= \{a\}\ .\ L_{nr},\ L_{nr} = \{b^j c^j:j \geq 0\}
			\end{align*}
			Como sabemos que $L_{nr}$ não é regular (é uma linguagem da forma $\{a^nb^n:n \geq 0\}$), chegamos em uma contradição, porque partindo de operações entre linguagens regulares obtivemos $L_{nr}$ como resultado. Então, concluímos que $L$ não é regular.
	\end{itemize}
	\begin{itemize}
		\item[(b)] Forneça um inteiro positivo p e mostre que a linguagem L satisfaz
as condições do Lema do Bombeamento para esse p.\\
		$Resp:$ Seja $p$ o inteiro fornecido pelo \textbf{Lema do Bombeamento} para $L$ e a palavra $w = a^pb^jc^j$. Tomando $p = 1,$ para $j \geq 0$. Com $x = a^0, y = a^p, p \geq |xy| \geq |y| \geq 1\ e\ z = b^j c^j$, considere todas as palavras que satisfazem ($i \geq 0$):
			\[xy^iz = a^0(a^{p})^i b^j c^j = a^{pi} b^j c^j\]
		Observamos que para todo inteiro $i,j \geq 0$ o \textbf{LB} é satisfeito. Então, para um inteiro $p = 1$, teremos um conjunto de palavras $W = a^{i} b^j c^j$, todas elas pertencendo à $L$.
	\end{itemize}
\end{document}
