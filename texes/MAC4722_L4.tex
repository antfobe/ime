% to change the appearance of the header, questions, problems or subproblems, see the homework.cls file or
% override the \Problem, \Subproblem, \question or \printtitle commands.

% The hidequestions option hides the questions. Remove it to print the questions in the text.

\title{MAC4722 - Lista 4}

\documentclass{homework}

\usepackage[utf8]{inputenc}
\usepackage{tikz}
\usetikzlibrary{automata,positioning}
% Set up your name, the course name and the homework set number.
\homeworksetup{
    username={Jean Fobe, N$^o$USP 7630573},
    course={Linguagens, Autômatos e Computabilidade - MAC4722},
    setnumber=4}
\begin{document}% this also prints the header.
\pagestyle{fancy}
\fancyfoot[L]{Jean Fobe, 7630573}

% use starred problems or subproblems to apply manual numbering.
\problem*{1}
\question{Seja $L \subseteq \Sigma^*$  uma linguagem regular.  Mostre que $L ^R$ também é regular, descrevendo como construir um autômato finito não determinístico para reconhecer $L^R$
, a partir de um autômato finito (determinístico ou não determinístico)
que reconhece $L$.}\\

	$Resp:$ Queremos mostrar que se $L$ é regular então $L^R$ também é regular.\\
	Seja $A = (Q_A,\Sigma_A,\delta_A,q_A,F_A)$ um autômato finito não determinístico satisfazendo $L = L(A)$. O autômato finito não determinístico $A^R$ definido abaixo aceita a linguagem $L^R$.
	\begin{enumerate}
		\item $A^R = (Q_A \cup \{q_s\},\Sigma_A,\delta_{A^R},q_s,\{q_A\})$, com $q_s \notin Q_A$;
		\item $p \in \delta_A(q,a) \iff q \in \delta_{A^R}(p,a)$, onde $a \in \Sigma_A$ e $p,q \in Q_A$.
	\end{enumerate}
	Com isso precisamos provar que existe uma sequencia de transições de q para p que é uma computação da palavra $\omega$ em $A$, se e somente se, existe uma computação de $\omega^R$ em $A^R$ de p para q, com $q,p \in Q_A$. A prova será feita por indução no comprimento de $\omega$.
(https://cs.stackexchange.com/questions/3251/how-to-show-that-a-reversed-regular-language-is-regular)


\problem*{2}
\question{Seja $\Sigma_3 = \left \{
				\begin{bmatrix} 0 \\ 0 \\ 0 \end{bmatrix},
				\begin{bmatrix} 0 \\ 0 \\ 1 \end{bmatrix},
				\begin{bmatrix} 0 \\ 1 \\ 0 \end{bmatrix},\ ...\ ,
				\begin{bmatrix} 1 \\ 1 \\ 1 \end{bmatrix}
				\right \}$.\\
		 $\Sigma_3$ contém todas as colunas de tamanho 3 formadas de 0s e 1s. Uma palavra de símbolos de $\Sigma_3$ é composta de 3 linhas de '0's e '1's. Considere cada linha como um número binário e seja $L = \{\omega \in \Sigma^3 :$ a terceira linha de $\omega$ é a soma das suas duas primeiras linhas $\}$.\\
		 Por exemplo,\\
		 \begin{changemargin}{1cm}{1cm} 
		 	$\begin{bmatrix} 0 \\ 0 \\ 1 \end{bmatrix}
		  	\begin{bmatrix} 1 \\ 0 \\ 0 \end{bmatrix}
		  	\begin{bmatrix} 1 \\ 1 \\ 0 \end{bmatrix} \in L, \quad mas \quad 
		  	\begin{bmatrix} 0 \\ 0 \\ 1 \end{bmatrix}
		  	\begin{bmatrix} 1 \\ 0 \\ 1 \end{bmatrix} \not\in L$
		 \end{changemargin}
		 Mostre que L é uma linguagem regular.\\
		 (Sugestão: Mostre que $L^R$ é uma linguagem regular. Depois, utilize o resultado do exercício anterior.)}
		 
\problem*{3}
\question{Seja L uma linguagem qualquer sobre $\Sigma$. Definimos $Remove(L)$ como
sendo a linguagem contendo todas as palavras que podem ser obtidas pela remoção
de um símbolo de uma palavra em L. Assim, $Remove(L) = \{xy : x \sigma y \in L,$ onde $x, y \in \Sigma^*$ e $\sigma \in \Sigma \}$. Mostre que a classe de linguagens regulares é fechada sob a operação
$Remove$.}

\problem*{4}
\question{Complete a prova do Teorema 5, mostrando que $L(C) = A \cup B$ (formalizando a ideia que vimos em aula).}

\problem*{5}
\question{Complete a prova do Teorema 6, mostrando que $L(C) = A B$ (formalizando a ideia que vimos em aula).}

\end{document}
